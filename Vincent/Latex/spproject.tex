\documentclass[10pt,a4paper]{article}
\usepackage[utf8]{inputenc}
\usepackage[english]{babel}
\usepackage[T1]{fontenc}
\usepackage{amsmath}
\usepackage{amsfonts}
\usepackage{amssymb}
\usepackage{subcaption}
\usepackage{makeidx}
\usepackage{graphicx}
\usepackage{fourier}
\usepackage{listings}
\usepackage{color}
\usepackage{hyperref}
\usepackage[left=2cm,right=2cm,top=2cm,bottom=2cm]{geometry}
\author{Felipe Bruno, Vincent Noculak}
\title{Space Physics Project}

\begin{document}

\maketitle
\newpage
\tableofcontents
\newpage


\section{Introduction}

In this project we are going to study the space weather of the northern hemisphere of the earth at the 6. January 2011 between 18 and 24 o'clock universal time. We are going to do this by analysing the data of the ACE, SuperDARN, AMPERE, Ground-based magnetometers and All-Sky Cameras.

The phenomenons of space weather are mostly driven by the solar wind and its IMF (interplanetary magnetic field). The most important phenomenon of the interaction of the IMF and the terrestrial magnetic field is the Dungey cycle. This cycle can only happen when there is a southward IMF. It consists of the following steps which take approximately 12 hours to complete.

It begins with magnetic reconnection at the magnetopause between the in the opposite direction pointing magnetic field lines of the IMF and the earth. Due to the magnetic tension force and the solar wind, the reconnected field lines get dragged into the tail of the earth. The adding of magnetic flux to the tail compresses the plasma sheet. Now magnetic reconnection occurs in the tail. The reconnected and now again closed field lines of the earth return to the dayside where the cycle can begin anew.

\section{Instruments}

\subsection{All-Sky Camera}

In our project we are going to use the data of All-Sky cameras located in Svalbard at Ny Ålesund. These instruments are used to investigate the occurrence of auroras against time.
 Using a fish-eye lens, All-Sky cameras are able to take images of the whole sky. With an optical filter they can target the characteristic green and red light of the aurora. The cameras we are using have the filters at 5577 Å and 6300 Å. The cameras also include a photon counter to measure the brightness to the auroras. At \ref{a1} a typical keogram using an All-Sky Camera can be seen. While the colour gives the brightness, the x-axis gives the time and the y-axis gives the elevation. The diagram is made by taking the middle column of pixels in every photo the camera has taken and lining them up in the right time order.

\begin{figure}[h]
	\includegraphics[scale = 0.20]{am-0024-5577.png}
	\centering
	\caption{Keogram of the All-Sky camera}
	\label{a1}
\end{figure}

\subsection{ACE}

The Advanced Composition Explorer, short ACE, is a satellite from NASA, which is among other things used to study the solar wind and its magnetic field. It orbits around the sun at the Lagrangian point $L_1$ so that it is always at a stable point between the earth and the sun, about $1.5 \cdot 10^6 km$ away from earth. 
One instrument of the satellite we are going to use is the SWEPAM (Solar Wind Electron, Proton and Alpha Monitor). With the instrument we can analysize the solar wind bulk speed to determine how much time it will need from the satellite to the earth. 
The other instrument we will use is the MAG (Magnetometer) to study the magnetic field of the solar wind.


\subsection{SuperDARN}

SuperDARN (Super Dual Auroral Radar Network) is a network of radars that consists of more than 30 low-power high frequency radars. It is used to measure plasma convection in the F-Region of the ionosphere at a high latitude. In \ref{s1} the area the radars of superDARN cover is shown. It can be seen that nearly the whole northern polar cap is covered by the radars. On Russia's side some ground at high latitude is not covered by the radars.
The radars are using the Doppler effect to measure convection. They send out electromagnetic waves at a specific frequency. In the Ionosphere the waves get reflected and have a Doppler shift due to movement of the plasma. From the change in frequency of the waves which return to the instruments, the velocity of the plasma can be determined.

\begin{figure}[h]
	\includegraphics[scale = 1.5]{sd_1.png}
	\centering
	\caption{Covered area of the SuperDARN}
	\label{s1}
\end{figure}

\subsection{AMPERE}

AMPERE stands for "Active Magnetosphere and Planetary Electrodynamics Response Experiment". It uses the over 60  Iridium satellites which are orbiting earth to get information about the space weather. On the satellites are magnetometers to measure magnetic fields. From the measurement of those fields, field-aligned currents can be derived, using Amperes law. Thus the currents drawn in the diagrams belonging to AMPERE are not measured directly. Because of the amount of iridium satellites, the measurements are always covering the whole earth.

\subsection{Ground-based magnetometers}

In this project we are going to analyse the data on Ground-based magnetometers. Most of the magnetometers we are using are located in Norway, Sweden or Finland. A current always generates a magnetic field (Ampere's law). Hence by measuring the magnetic field on the ground, currents in the Ionosphere can be measured indirectly.

\section{Observations}

\subsection{All-Sky Camera}

The figure \ref{fig:sfig1} and \ref{fig:sfig2} show the keograms of the All-Sky Camera for the whole day on the 06.01.2011 for the green and red aurora. We look at the time span from 18 to 24 UT. On the first sight it can be seen that before 22 UT there is nearly no aurora, while after 22 UT there is a big onset of aurora. This timespan, 22-24 UT, will be discussed the most in this project.

Between 18 UT and 19 UT (figure \ref{fig:sfig3} and \ref{fig:sfig4}) it can be seen, that the camera detects a lot of light at low latitudes. Most of this bright structure is outside of the range of the imager and is very stable in time. Hence it is difficult to decide if it is an aurora or a cloud. At 18:45 UT there is a bright source at higher latitude, which moves southward for 10 minutes. By looking at the pictures the camera took separately and comparing the structure at both wavelengths, it can be seen that this is probably a cloud, which moves southward and then dissolves. 
In the timespan 19-22 UT, no auroral activity is happening in the area the All-sky camera observes.

 In figure \ref{fig:sfig5}, \ref{fig:sfig6} and \ref{i1} it can be seen, that, starting with 22:20 UT, a long aurora forms at the western sky up to 82° geomagnetic latitude and moves southward down to 75° MLAT at 22:50 UT. During the end of that movement, starting from 22:45 UT, there is a large onset of aurora, which starts at low latitudes (72.3 MLAT). This onset then moves very quickly to higher latitudes. At 23:28 the auroras are already at 80° MLAT and stay at latitudes this high up until 0:00 UT. During that movement, the auroras a very unstable, for example when they increase their intensity a lot at 23:10 UT, then get less intense and peak again from 23:25 to 23:33 UT. The last peak of onset of auroras then is from 23:50 to after 0:00 UT.
 Also, starting from 22:10 auroras at low latitudes around 70° MLAT can be seen. Those auroras are the ones, that brighten at the big onset of aurora at 22:45 UT.
The biggest onset of green aurora is between 23:28 and 23:29 UT. In this time frame the green aurora covers nearly the whole sky, while the red is not as strong. The red aurora is especially strong between 23:03 and 23:09 UT.
The sudden strong increases of auroral activity starting from 22:45 UT are a sign of a starting expansion phase of a substorm.

 In the night of the 06.01.2011 it looks like there is a especially big substorm happening between 22:20 and 24:00 UT, which has three peeks in activity at 23:05 UT, 23:28 UT and between 23:50-24:00 UT. At 22:45 the expansion phase of this substorm starts. It has the typical characteristics of this phase, because the auroral arc suddenly brightens, moves poleward quickly and then already fills nearly the whole sky at 23:28 UT (figure \ref{i2}). In this phase, the auroras move from 70° MLAT at 22:40 UT to 81°-82.5° MLAT at 23:32 UT and stay at this latitude to after 24:00 UT. It happens that the intensity of auroral activity decreases between 23:05 and 23:20 UT and 23:45 and 23:50 UT, but we would not describe this time-spans as a recovery phase, because they last a very short time and are quickly followed by a big increase of auroral activity again. The real recovery phase of the substorm begins after 24:00 UT.   

\begin{figure}[h]
	\begin{subfigure}{.5\textwidth}
		\centering
		\includegraphics[width=.8\linewidth]{am-0024-5577.png}
		\caption{0-24 UT, 5577 Å}
		\label{fig:sfig1}
	\end{subfigure}
	\begin{subfigure}{.5\textwidth}
		\centering
		\includegraphics[width=.8\linewidth]{am-0024-6300.png}
		\caption{0-24 UT, 6300 Å}
		\label{fig:sfig2}
	\end{subfigure}
	\begin{subfigure}{.5\textwidth}
		\centering
		\includegraphics[width=.8\linewidth]{am-1819-5577.png}
		\caption{18-19 UT, 5577 Å}
		\label{fig:sfig3}
	\end{subfigure}
	\begin{subfigure}{.5\textwidth}
		\centering
		\includegraphics[width=.8\linewidth]{am-1819-6300.png}
		\caption{18-19 UT, 6300 Å}
		\label{fig:sfig4}
	\end{subfigure}
	\begin{subfigure}{.5\textwidth}
		\centering
		\includegraphics[width=.8\linewidth]{am-2223-5577.png}
		\caption{22-23 UT, 5577 Å}
		\label{fig:sfig5}
	\end{subfigure}
	\begin{subfigure}{.5\textwidth}
		\centering
		\includegraphics[width=.8\linewidth]{am-2223-6300.png}
		\caption{22-23 UT, 6300 Å}
		\label{fig:sfig6}
	\end{subfigure}
	\begin{subfigure}{.5\textwidth}
		\centering
		\includegraphics[width=.8\linewidth]{am-2324-5577.png}
		\caption{23-24 UT, 5577 Å}
		\label{fig:sfig7}
	\end{subfigure}
	\begin{subfigure}{.5\textwidth}
		\centering
		\includegraphics[width=.8\linewidth]{am-2324-6300.png}
		\caption{23-24 UT, 6300 Å}
		\label{fig:sfig8}
	\end{subfigure}
	\caption{Keograms of the All-sky camera for different times and wavelengths }
	\label{fig:fig}
\end{figure}

\begin{figure}
\begin{subfigure}{.5\textwidth}
	\includegraphics[width=.8\linewidth]{ame-2228-6300.jpg}
	\centering
	\caption{22:28 UT, $\lambda$ = 6300 Å}
	\label{i1}
\end{subfigure}
\begin{subfigure}{.5\textwidth}
	\includegraphics[width=.8\linewidth]{ame-2328-5577.jpg}
	\centering
	\caption{23:28 UT, $\lambda$ = 5577 Å}
	\label{i2}
\end{subfigure}
\caption{Images of the All-sky camera}
\end{figure}


\subsection{ACE}

I figure \ref{ace1} the components of the magnetic field of the solar wind can be seen in the timespan 17:00-23:10 UT. In the coordinate system, we are using, the geocentric solar magnetospheric system (GSM), it only depends on the z-component of the magnetic field, if magnetic reconnection at the magnetopause is possible. Is the z-component of the magnetic field of the solar wind lower than zero, it has the opposite sign than the terrestrial magnetic field. Thus magnetic reconnection is possible. 

It can be observed, that between 17:00 and 20:10 UT, most of the time $B_Z$ is bigger than zero. Hence magnetic reconnection is nearly all this time not possible at the magnetopause.
At 20:10 UT $B_Z$ drops more than $20 nT$ and then stays negative nearly all the time. Just at 22:00 UT and around 21:20 UT it goes positive for a very short amount of time. Thus from the time, the solar wind, which passes the ACE at 20:10 UT, reaches the earth, magnetic reconnection is possible. 

In our coordinate system, the x-axis is parallel to the connection line of earth and sun. Thus we only need to look at the solar wind velocity in x-direction to calculate, how long the wind need from the satellite to earth.
In the timespan 17:00-20:00 UT the velocity $v_x$ of the solar wind is approximately constant at $v_x = 350 \frac{km}{s}$ (figure \ref{ace3}). In this time the distance between satellite and earth, $s_x$, is between $1,4509 \cdot 10^6 km$ and $1,4512 \cdot 10^6 km$. Thus the solar wind need $t = \frac{1,4510 \cdot 10^6 km}{350 \frac{km}{s}} = 69 min$ to the earth.

At 20:10 UT, when the magnetic field $B_Z$ of the solar wind gets negative, the solar wind has a velocity of $(370 \pm 10) \frac{km}{s}$ and the distance between satellite and earth is $(1.4513 \pm 0.0001) \cdot 10^6 km$. With this values, we can calculate the time, the solar wind from 20:10 needs to reach the earth. We use the approximation, that the solar wind stays at the same velocity between earth and satellite. The time the wind needs is $t = \frac{1.4513 \cdot 10^6 km}{370 \frac{km}{s}} = 65 min$. Thus it needs $65 \pm 2 min$ and arrives at the earth at $21:13 - 21:17 UT$. From this time, magnetic reconnection can happen at the magnetopause.





\begin{figure}[h]
	\begin{subfigure}[h]{.5\textwidth}
		\centering
		\includegraphics[width=.8\linewidth]{ace-17-2310-b.png}
		\caption{Magnetic field of the solar wind, 17:00-23:10 UT}
		\label{ace1}
	\end{subfigure}
	\begin{subfigure}[h]{.5\textwidth}
		\centering
		\includegraphics[width=.8\linewidth]{ace-17-2310-v-s.png}
		\caption{Velocity of the solar wind and position of the satellite, 17:00-23:10 UT}
		\label{ace3}
	\end{subfigure}
	\caption{Measurements of the ACE}
	\label{ace}
\end{figure}







\section{Sources}

Websites:

	\url{http://tid.uio.no/plasma/aurora/tech.html}
	
	\url{https://en.wikipedia.org/wiki/Advanced_Composition_Explorer}
	
	\url{https://en.wikipedia.org/wiki/Super_Dual_Auroral_Radar_Network}
	
	\url{https://en.wikipedia.org/wiki/Iridium_satellite_constellation}
	
	\url{http://www.jhuapl.edu/newscenter/pressreleases/2010/100818.asp}
	
	\url{https://directory.eoportal.org/web/eoportal/satellite-missions/a/ampere}
	
Graphics:

	\url{http://tid.uio.no/plasma/aurora/data.php}
	
	\url{http://vt.superdarn.org/tiki-index.php?page=radarFoV}


\end{document}